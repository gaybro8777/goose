\chapter{General purpose tools}
\label{seq}

\begin{enumerate}
\item \texttt{goose-comparativemap}%: visualisation of comparative maps. It builds a image given specific patterns between two sequences.
\item \texttt{goose-BruteForceString}%: it generates, line by line, multiple combinations of strings up to a certain size.
\item \texttt{goose-char2line}%: it transforms each char into a char in each line.
\item \texttt{goose-sum}%: it adds the second column value to the first column value.
\item \texttt{goose-min}%: it finds the minimum value between two column values.
\item \texttt{goose-minus}%: it substracts the second column value to the first column value.
\item \texttt{goose-max}%: it finds the mmaximum value between two column values.
\item \texttt{goose-extract}%: it extracts a subsequence of a sequence by coordinates.
\item \texttt{goose-segment}%: it segments a sequence given a certain threshold.
\item \texttt{goose-reverse}: it reverses the order of a sequence.
\end{enumerate}

%TO DO
%\section{Program goose-comparativemap}
The \texttt{goose-comparativemap} ...

For help type:
\begin{lstlisting}
./goose-comparativemap -h
\end{lstlisting}
In the following subsections, we explain the input and output paramters.

\subsection*{Input parameters}

The \texttt{goose-comparativemap} program needs ...
The attribution is given according to:
\begin{lstlisting}
TO DO
\end{lstlisting}

An example on such an input file is:
\begin{lstlisting}
TO DO
\end{lstlisting}

\subsection*{Output}
The output of the \texttt{goose-comparativemap} program ...
An example, for the input, is:
\begin{lstlisting}
TO DO
\end{lstlisting}
%\section{Program goose-BruteForceString}
The \texttt{goose-BruteForceString} ...

For help type:
\begin{lstlisting}
./goose-BruteForceString -h
\end{lstlisting}
In the following subsections, we explain the input and output paramters.

\subsection*{Input parameters}

The \texttt{goose-BruteForceString} program needs ...
The attribution is given according to:
\begin{lstlisting}
TO DO
\end{lstlisting}

An example on such an input file is:
\begin{lstlisting}
TO DO
\end{lstlisting}

\subsection*{Output}
The output of the \texttt{goose-BruteForceString} program ...
An example, for the input, is:
\begin{lstlisting}
TO DO
\end{lstlisting}
%\section{Program goose-char2line}
The \texttt{goose-char2line} splits a sequence into lines, creating an output sequence which has a char for each line.

For help type:
\begin{lstlisting}
./goose-char2line -h
\end{lstlisting}
In the following subsections, we explain the input and output paramters.

\subsection*{Input parameters}

The \texttt{goose-char2line} program needs two streams for the computation,
namely the input and output standard. The input stream is a sequence file.\\
The attribution is given according to:
\begin{lstlisting}
Usage: ./goose-char2line [options] [[--] args]
   or: ./goose-char2line [options]

It splits a sequence into lines, creating an output sequence which has a char for each line.

    -h, --help        show this help message and exit

Basic options
    < input.seq       Input sequence file (stdin)
    > output.seq      Output sequence file (stdout)

Example: ./goose-char2line < input.seq > output.seq
\end{lstlisting}
An example on such an input file is:
\begin{lstlisting}
ACAAGACGGCCTCCTGCTGCTGCTGCTCTCCGGGGCCACGGCCCTGGAGGGTCCACCGCTGCCCTGCTGCCATTGTCCCC
GGCCCCACCTAAGGAAAAGCAGCCTCCTGACTTTCCTCGCTTGGGCCGAGACAGCGAGCATATGCAGGAAGCGGCAGGAA
GTGGTTTGAGTGGACCTCCGGGCCCCTCATAGGAGAGGAAGCTCGGGAGGTGGCCAGGCGGCAGGAAGCAGGCCAGTGCC
GCGAATCCGCGCGCCGGGACAGAATCTCCTGCAAAGCCCTGCAGGAACTTCTTCTGGAAGACCTTCTCCACCCCCCCAGC
TAAAACCTCACCCATGAATGCTCACGCAAGTTTAATTACAGACCTGAAACAAGATGCCATTGTCCCCCGGCCTCCTGCTG
CTGCTGCTCTCCGGGGCCACGGCCACCGCTGCCCTGCCCCTGGAGGGTGGCCCCACCGGCCGAGACAGCGAGCATATGCA
GGAAGCGGCAGGAATAAGGAAAAGCAGCCTCCTGACTTTCCTCGCTTGGTGGTTTGAGTGGACCTCCCAGGCCAGTGCCG
GGCCCCTCATAGGAGAGGAAGCTCGGGAGGTGGCCAGGCGGCAGGAAGGCGCACCCCCCCAGCAATCCGCGCGCCGGGAC
AGAATGCCCTGCAGGAACTTCTTCTGGAAGACCTTCTCCTCCTGCAAATAAAACCTCACCCATGAATGCTCACGCAAGTT
TAATTACAGACCTGAA
\end{lstlisting}

\subsection*{Output}
The output of the \texttt{goose-char2line} program is a group sequence splited by \textbackslash n foreach character.\\
An example, for the input, is:
\begin{lstlisting}
A
C
A
A
G
A
C
G
G
C
C
T
C
C
T
G
C
T
G
C
T
...
\end{lstlisting}
%\section{Program goose-sum}
The \texttt{goose-sum} ...

For help type:
\begin{lstlisting}
./goose-sum -h
\end{lstlisting}
In the following subsections, we explain the input and output paramters.

\subsection*{Input parameters}

The \texttt{goose-sum} program needs ...
The attribution is given according to:
\begin{lstlisting}
TO DO
\end{lstlisting}

An example on such an input file is:
\begin{lstlisting}
TO DO
\end{lstlisting}

\subsection*{Output}
The output of the \texttt{goose-sum} program ...
An example, for the input, is:
\begin{lstlisting}
TO DO
\end{lstlisting}
%\section{Program goose-min}
The \texttt{goose-min} ...

For help type:
\begin{lstlisting}
./goose-min -h
\end{lstlisting}
In the following subsections, we explain the input and output paramters.

\subsection{Input parameters}

The \texttt{goose-min} program needs ...
The attribution is given according to:
\begin{lstlisting}
TO DO
\end{lstlisting}

An example on such an input file is:
\begin{lstlisting}
TO DO
\end{lstlisting}

\subsection{Output}
The output of the \texttt{goose-min} program ...
An example, for the input, is:
\begin{lstlisting}
TO DO
\end{lstlisting}
%\section{Program goose-minus}
The \texttt{goose-minus} ...

For help type:
\begin{lstlisting}
./goose-minus -h
\end{lstlisting}
In the following subsections, we explain the input and output paramters.

\subsection*{Input parameters}

The \texttt{goose-minus} program needs ...
The attribution is given according to:
\begin{lstlisting}
TO DO
\end{lstlisting}

An example on such an input file is:
\begin{lstlisting}
TO DO
\end{lstlisting}

\subsection*{Output}
The output of the \texttt{goose-minus} program ...
An example, for the input, is:
\begin{lstlisting}
TO DO
\end{lstlisting}
%\section{Program goose-max}
The \texttt{goose-max} ...

For help type:
\begin{lstlisting}
./goose-max -h
\end{lstlisting}
In the following subsections, we explain the input and output paramters.

\subsection{Input parameters}

The \texttt{goose-max} program needs ...
The attribution is given according to:
\begin{lstlisting}
TO DO
\end{lstlisting}

An example on such an input file is:
\begin{lstlisting}
TO DO
\end{lstlisting}

\subsection{Output}
The output of the \texttt{goose-max} program ...
An example, for the input, is:
\begin{lstlisting}
TO DO
\end{lstlisting}
%\section{Program goose-extract}
The \texttt{goose-extract} ...

For help type:
\begin{lstlisting}
./goose-extract -h
\end{lstlisting}
In the following subsections, we explain the input and output paramters.

\subsection{Input parameters}

The \texttt{goose-extract} program needs ...
The attribution is given according to:
\begin{lstlisting}
TO DO
\end{lstlisting}

An example on such an input file is:
\begin{lstlisting}
TO DO
\end{lstlisting}

\subsection{Output}
The output of the \texttt{goose-extract} program ...
An example, for the input, is:
\begin{lstlisting}
TO DO
\end{lstlisting}
%\section{Program goose-segment}
The \texttt{goose-segment} ...

For help type:
\begin{lstlisting}
./goose-segment -h
\end{lstlisting}
In the following subsections, we explain the input and output paramters.

\subsection{Input parameters}

The \texttt{goose-segment} program needs ...
The attribution is given according to:
\begin{lstlisting}
TO DO
\end{lstlisting}

An example on such an input file is:
\begin{lstlisting}
TO DO
\end{lstlisting}

\subsection{Output}
The output of the \texttt{goose-segment} program ...
An example, for the input, is:
\begin{lstlisting}
TO DO
\end{lstlisting}
\section{Program goose-reverse}
The \texttt{goose-reverse} reverses the order of a sequence file.

For help type:
\begin{lstlisting}
./goose-reverse -h
\end{lstlisting}
In the following subsections, we explain the input and output paramters.

\subsection*{Input parameters}

The \texttt{goose-reverse} program needs two streams for the computation,
namely the input and output standard. The input stream is a sequence file.\\
The attribution is given according to:
\begin{lstlisting}
Usage: ./goose-reverse [options] [[--] args]
   or: ./goose-reverse [options]

It reverses the order of a sequence file.

    -h, --help        show this help message and exit

Basic options
    < input.seq       Input sequence file (stdin)
    > output.seq      Output sequence file (stdout)

Example: ./goose-reverse < input.seq > output.seq
\end{lstlisting}
An example on such an input file is:
\begin{lstlisting}
ACAAGACGGCCTCCTGCTGCTGCTGCTCTCCGGGGCCACGGCCCTGGAGGGTCCACCGCTGCCCTGCTGCCATTGTCCCC
GGCCCCACCTAAGGAAAAGCAGCCTCCTGACTTTCCTCGCTTGGGCCGAGACAGCGAGCATATGCAGGAAGCGGCAGGAA
GTGGTTTGAGTGGACCTCCGGGCCCCTCATAGGAGAGGAAGCTCGGGAGGTGGCCAGGCGGCAGGAAGCAGGCCAGTGCC
GCGAATCCGCGCGCCGGGACAGAATCTCCTGCAAAGCCCTGCAGGAACTTCTTCTGGAAGACCTTCTCCACCCCCCCAGC
TAAAACCTCACCCATGAATGCTCACGCAAGTTTAATTACAGACCTGAAACAAGATGCCATTGTCCCCCGGCCTCCTGCTG
CTGCTGCTCTCCGGGGCCACGGCCACCGCTGCCCTGCCCCTGGAGGGTGGCCCCACCGGCCGAGACAGCGAGCATATGCA
GGAAGCGGCAGGAATAAGGAAAAGCAGCCTCCTGACTTTCCTCGCTTGGTGGTTTGAGTGGACCTCCCAGGCCAGTGCCG
GGCCCCTCATAGGAGAGGAAGCTCGGGAGGTGGCCAGGCGGCAGGAAGGCGCACCCCCCCAGCAATCCGCGCGCCGGGAC
AGAATGCCCTGCAGGAACTTCTTCTGGAAGACCTTCTCCTCCTGCAAATAAAACCTCACCCATGAATGCTCACGCAAGTT
TAATTACAGACCTGAA
\end{lstlisting}

\subsection*{Output}
The output of the \texttt{goose-reverse} program is a group sequence.\\
An example, for the input, is:
\begin{lstlisting}
AAGTCCAGACATTAATTTGAACGCACTCGTAAGTACCCACTCCAAAATAAACGTCCTCCTCTTCCAGAAGGTCTTCTTCA
AGGACGTCCCGTAAGACAGGGCCGCGCGCCTAACGACCCCCCCACGCGGAAGGACGGCGGACCGGTGGAGGGCTCGAAGG
AGAGGATACTCCCCGGGCCGTGACCGGACCCTCCAGGTGAGTTTGGTGGTTCGCTCCTTTCAGTCCTCCGACGAAAAGGA
ATAAGGACGGCGAAGGACGTATACGAGCGACAGAGCCGGCCACCCCGGTGGGAGGTCCCCGTCCCGTCGCCACCGGCACC
GGGGCCTCTCGTCGTCGTCGTCCTCCGGCCCCCTGTTACCGTAGAACAAAGTCCAGACATTAATTTGAACGCACTCGTAA
GTACCCACTCCAAAATCGACCCCCCCACCTCTTCCAGAAGGTCTTCTTCAAGGACGTCCCGAAACGTCCTCTAAGACAGG
GCCGCGCGCCTAAGCGCCGTGACCGGACGAAGGACGGCGGACCGGTGGAGGGCTCGAAGGAGAGGATACTCCCCGGGCCT
CCAGGTGAGTTTGGTGAAGGACGGCGAAGGACGTATACGAGCGACAGAGCCGGGTTCGCTCCTTTCAGTCCTCCGACGAA
AAGGAATCCACCCCGGCCCCTGTTACCGTCGTCCCGTCGCCACCTGGGAGGTCCCGGCACCGGGGCCTCTCGTCGTCGTC
GTCCTCCGGCAGAACA
\end{lstlisting}