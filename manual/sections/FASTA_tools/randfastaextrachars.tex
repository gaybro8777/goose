\section{Program goose-randfastaextrachars}
The \texttt{goose-randfastaextrachars} substitues in the DNA sequence the outside ACGT chars by random ACGT symbols. It works both in FASTA and Multi-FASTA file formats.\\
For help type:
\begin{lstlisting}
./goose-randfastaextrachars -h
\end{lstlisting}
In the following subsections, we explain the input and output paramters.

\subsection*{Input parameters}

The \texttt{goose-randfastaextrachars} program needs two streams for the computation,
namely the input and output standard. The input stream is a FASTA or Multi-FASTA file.\\
The attribution is given according to:
\begin{lstlisting}
Usage: ./goose-randfastaextrachars [options] [[--] args]
   or: ./goose-randfastaextrachars [options]

It substitues in the DNA sequence the outside ACGT chars by random ACGT symbols.
It works both in FASTA and Multi-FASTA file formats


    -h, --help            show this help message and exit

Basic options
    < input.fasta         Input FASTA or Multi-FASTA file format (stdin)
    > output.fasta        Output FASTA or Multi-FASTA file format (stdout)

Example: ./goose-randfastaextrachars < input.fasta > output.fasta
\end{lstlisting}

An example on such an input file is:
\begin{lstlisting}
to do
\end{lstlisting}

\subsection*{Output}
The output of the \texttt{goose-randfastaextrachars} program is a FASTA or Multi-FASTA file.\\
An example, for the input, is:
\begin{lstlisting}
to do
\end{lstlisting}