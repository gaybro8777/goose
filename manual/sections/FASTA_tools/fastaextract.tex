\section{Program goose-fastaextract}
The \texttt{goose-fastaextract} extracts sequences from a FASTA file, which the range is defined by the user in the parameters.\\
For help type:
\begin{lstlisting}
./goose-fastaextract -h
\end{lstlisting}
In the following subsections, we explain the input and output paramters.

\subsection{Input parameters}

The \texttt{goose-fastaextract} program needs two paramenters, which defines the begin and the end of the extraction, and two streams for the computation, namely the input and output standard. The input stream is a FASTA file.\\
The attribution is given according to:
\begin{lstlisting}
Usage: ./goose-fastaextract -i <init> -e <end> < input.fasta > out.seq
It extracts sequences from a FASTA file.
\end{lstlisting}
An example on such an input file is:
\begin{lstlisting}
>AB000264 |acc=AB000264|descr=Homo sapiens mRNA 
ACAAGACGGCCTCCTGCTGCTGCTGCTCTCCGGGGCCACGGCCCTGGAGGGTCCACCGCTGCCCTGCTGCCATTGTCCCC
GGCCCCACCTAAGGAAAAGCAGCCTCCTGACTTTCCTCGCTTGGGCCGAGACAGCGAGCATATGCAGGAAGCGGCAGGAA
GTGGTTTGAGTGGACCTCCGGGCCCCTCATAGGAGAGGAAGCTCGGGAGGTGGCCAGGCGGCAGGAAGCAGGCCAGTGCC
GCGAATCCGCGCGCCGGGACAGAATCTCCTGCAAAGCCCTGCAGGAACTTCTTCTGGAAGACCTTCTCCACCCCCCCAGC
TAAAACCTCACCCATGAATGCTCACGCAAGTTTAATTACAGACCTGAA
\end{lstlisting}

\subsection{Output}
The output of the \texttt{goose-fastaextract} program is a group sequence.\\
An example, using the value 0 as extraction starting point and the 50 as the end, for the provided input, is:
\begin{lstlisting}
ACAAGACGGCCTCCTGCTGCTGCTGCTCTCCGGGGCCACGGCCCTGGAGG
\end{lstlisting}