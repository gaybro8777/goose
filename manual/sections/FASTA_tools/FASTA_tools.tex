\chapter{FASTA tools}
\label{fasta}
%Write something about FASTA tools

Current available FASTA tools, for analysis and manipulation, are:
\begin{enumerate}
\item \texttt{goose-fasta2seq}: it converts a FASTA or Multi-FASTA file format to a seq.
\item \texttt{goose-fastaextract}: it extracts sequences from a FASTA file, which the range is defined by the user in the parameters.
\item \texttt{goose-fastaextractbyread}: it extracts sequences from each read in a Multi-FASTA file (splited by \textbackslash n), which the range is defined by the user in the parameters.
\item \texttt{goose-fastainfo}: it shows the readed information of a FASTA or Multi-FASTA file format.
\item \texttt{goose-mutatefasta}: it reates a synthetic mutation of a fasta file given specific rates of editions, deletions and additions.
\item \texttt{goose-randfastaextrachars}: it substitues in the DNA sequence the outside ACGT chars by random ACGT symbols.
\item \texttt{goose-extractreadbypattern}: it extracts reads from a Multi-FASTA file format given a pattern in the header.
\item \texttt{goose-findnpos}: it reports the ''N'' regions in a sequence or FASTA (seq) file.
\item \texttt{goose-seq2fasta}: it converts a genomic sequence to pseudo FASTA file format.
\item \texttt{goose-permuteseqbyblocks}
\item \texttt{goose-splitreads}: it splits a Multi-FASTA file to multiple FASTA files.

%Incomplete tools
%%\item \texttt{goose-reverselm}%: it reverses the order of a large sequence. Low memory usage for large files.
\end{enumerate}

\section{Program goose-fasta2seq}
The \texttt{goose-fasta2seq} converts a FASTA or Multi-FASTA file format to a seq.\\
For help type:
\begin{lstlisting}
./goose-fasta2seq -h
\end{lstlisting}
In the following subsections, we explain the input and output paramters.

\subsection{Input parameters}

The \texttt{goose-fasta2seq} program needs two streams for the computation,
namely the input and output standard. The input stream is a FASTA or Multi-FASTA file.\\
The attribution is given according to:
\begin{lstlisting}
Usage: ./goose-fasta2seq [options] [[--] args]
   or: ./goose-fasta2seq [options]

It converts a FASTA or Multi-FASTA file format to a seq.

    -h, --help            show this help message and exit

Basic options
    < input.fasta         Input FASTA or Multi-FASTA file format (stdin)
    > output.seq          Output sequence file (stdout)

Example: ./goose-fasta2seq < input.fasta > output.seq
\end{lstlisting}
An example on such an input file is:
\begin{lstlisting}
>AB000264 |acc=AB000264|descr=Homo sapiens mRNA 
ACAAGACGGCCTCCTGCTGCTGCTGCTCTCCGGGGCCACGGCCCTGGAGGGTCCACCGCTGCCCTGCTGCCATTGTCCCC
GGCCCCACCTAAGGAAAAGCAGCCTCCTGACTTTCCTCGCTTGGGCCGAGACAGCGAGCATATGCAGGAAGCGGCAGGAA
GTGGTTTGAGTGGACCTCCGGGCCCCTCATAGGAGAGGAAGCTCGGGAGGTGGCCAGGCGGCAGGAAGCAGGCCAGTGCC
GCGAATCCGCGCGCCGGGACAGAATCTCCTGCAAAGCCCTGCAGGAACTTCTTCTGGAAGACCTTCTCCACCCCCCCAGC
TAAAACCTCACCCATGAATGCTCACGCAAGTTTAATTACAGACCTGAA
>AB000263 |acc=AB000263|descr=Homo sapiens mRNA 
ACAAGATGCCATTGTCCCCCGGCCTCCTGCTGCTGCTGCTCTCCGGGGCCACGGCCACCGCTGCCCTGCCCCTGGAGGGT
GGCCCCACCGGCCGAGACAGCGAGCATATGCAGGAAGCGGCAGGAATAAGGAAAAGCAGCCTCCTGACTTTCCTCGCTTG
GTGGTTTGAGTGGACCTCCCAGGCCAGTGCCGGGCCCCTCATAGGAGAGGAAGCTCGGGAGGTGGCCAGGCGGCAGGAAG
GCGCACCCCCCCAGCAATCCGCGCGCCGGGACAGAATGCCCTGCAGGAACTTCTTCTGGAAGACCTTCTCCTCCTGCAAA
TAAAACCTCACCCATGAATGCTCACGCAAGTTTAATTACAGACCTGAA
\end{lstlisting}

\subsection{Output}

The output of the \texttt{goose-fasta2seq} program is a group sequence. \\
An example, for the input, is:
\begin{lstlisting}
ACAAGACGGCCTCCTGCTGCTGCTGCTCTCCGGGGCCACGGCCCTGGAGGGTCCACCGCTGCCCTGCTGCCATTGTCCCC
GGCCCCACCTAAGGAAAAGCAGCCTCCTGACTTTCCTCGCTTGGGCCGAGACAGCGAGCATATGCAGGAAGCGGCAGGAA
GTGGTTTGAGTGGACCTCCGGGCCCCTCATAGGAGAGGAAGCTCGGGAGGTGGCCAGGCGGCAGGAAGCAGGCCAGTGCC
GCGAATCCGCGCGCCGGGACAGAATCTCCTGCAAAGCCCTGCAGGAACTTCTTCTGGAAGACCTTCTCCACCCCCCCAGC
TAAAACCTCACCCATGAATGCTCACGCAAGTTTAATTACAGACCTGAAACAAGATGCCATTGTCCCCCGGCCTCCTGCTG
CTGCTGCTCTCCGGGGCCACGGCCACCGCTGCCCTGCCCCTGGAGGGTGGCCCCACCGGCCGAGACAGCGAGCATATGCA
GGAAGCGGCAGGAATAAGGAAAAGCAGCCTCCTGACTTTCCTCGCTTGGTGGTTTGAGTGGACCTCCCAGGCCAGTGCCG
GGCCCCTCATAGGAGAGGAAGCTCGGGAGGTGGCCAGGCGGCAGGAAGGCGCACCCCCCCAGCAATCCGCGCGCCGGGAC
AGAATGCCCTGCAGGAACTTCTTCTGGAAGACCTTCTCCTCCTGCAAATAAAACCTCACCCATGAATGCTCACGCAAGTT
TAATTACAGACCTGAA
\end{lstlisting}

\section{Program goose-fastaextract}
The \texttt{goose-fastaextract} extracts sequences from a FASTA file, which the range is defined by the user in the parameters.\\
For help type:
\begin{lstlisting}
./goose-fastaextract -h
\end{lstlisting}
In the following subsections, we explain the input and output paramters.

\subsection{Input parameters}

The \texttt{goose-fastaextract} program needs two paramenters, which defines the begin and the end of the extraction, and two streams for the computation, namely the input and output standard. The input stream is a FASTA file.\\
The attribution is given according to:
\begin{lstlisting}
Usage: ./goose-fastaextract [options] [[--] args]
   or: ./goose-fastaextract [options]

It extracts sequences from a FASTA file.

    -h, --help            show this help message and exit

Basic options
    -i, --init=<int>      The first position to start the extraction (default 0)
    -e, --end=<int>       The last extract position (default 100)
    < input.fasta         Input FASTA or Multi-FASTA file format (stdin)
    > output.seq          Output sequence file (stdout)

Example: ./goose-fastaextract -i <init> -e <end> < input.fasta > output.seq
\end{lstlisting}
An example on such an input file is:
\begin{lstlisting}
>AB000264 |acc=AB000264|descr=Homo sapiens mRNA 
ACAAGACGGCCTCCTGCTGCTGCTGCTCTCCGGGGCCACGGCCCTGGAGGGTCCACCGCTGCCCTGCTGCCATTGTCCCC
GGCCCCACCTAAGGAAAAGCAGCCTCCTGACTTTCCTCGCTTGGGCCGAGACAGCGAGCATATGCAGGAAGCGGCAGGAA
GTGGTTTGAGTGGACCTCCGGGCCCCTCATAGGAGAGGAAGCTCGGGAGGTGGCCAGGCGGCAGGAAGCAGGCCAGTGCC
GCGAATCCGCGCGCCGGGACAGAATCTCCTGCAAAGCCCTGCAGGAACTTCTTCTGGAAGACCTTCTCCACCCCCCCAGC
TAAAACCTCACCCATGAATGCTCACGCAAGTTTAATTACAGACCTGAA
\end{lstlisting}

\subsection{Output}
The output of the \texttt{goose-fastaextract} program is a group sequence.\\
An example, using the value 0 as extraction starting point and the 50 as the end, for the provided input, is:
\begin{lstlisting}
ACAAGACGGCCTCCTGCTGCTGCTGCTCTCCGGGGCCACGGCCCTGGAGG
\end{lstlisting}
\section{Program goose-fastaextractbyread}
The \texttt{goose-fastaextractbyread} extracts sequences from a FASTA or Multi-FASTA file, which the range is defined by the user in the parameters.\\
For help type:
\begin{lstlisting}
./goose-fastaextractbyread -h
\end{lstlisting}
In the following subsections, we explain the input and output paramters.

\subsection*{Input parameters}

The \texttt{goose-fastaextractbyread} program needs two paramenters, which defines the begin and the end of the extraction, and two streams for the computation, namely the input and output standard. The input stream is a FASTA or Multi-FASTA file.\\
The attribution is given according to:
\begin{lstlisting}
Usage: ./goose-fastaextractbyread [options] [[--] args]
   or: ./goose-fastaextractbyread [options]

It extracts sequences from each read in a Multi-FASTA file (splited by \n)

    -h, --help            show this help message and exit

Basic options
    -i, --init=<int>      The first position to start the extraction (default 0)
    -e, --end=<int>       The last extract position (default 100)
    < input.fasta         Input FASTA or Multi-FASTA file format (stdin)
    > output.fasta        Output FASTA or Multi-FASTA file format (stdout)

Example: ./goose-fastaextractbyread -i <init> -e <end> < input.fasta > output.fasta
\end{lstlisting}
An example on such an input file is:
\begin{lstlisting}
>AB000264 |acc=AB000264|descr=Homo sapiens mRNA 
ACAAGACGGCCTCCTGCTGCTGCTGCTCTCCGGGGCCACGGCCCTGGAGGGTCCACCGCTGCCCTGCTGCCATTGTCCCC
GGCCCCACCTAAGGAAAAGCAGCCTCCTGACTTTCCTCGCTTGGGCCGAGACAGCGAGCATATGCAGGAAGCGGCAGGAA
GTGGTTTGAGTGGACCTCCGGGCCCCTCATAGGAGAGGAAGCTCGGGAGGTGGCCAGGCGGCAGGAAGCAGGCCAGTGCC
GCGAATCCGCGCGCCGGGACAGAATCTCCTGCAAAGCCCTGCAGGAACTTCTTCTGGAAGACCTTCTCCACCCCCCCAGC
TAAAACCTCACCCATGAATGCTCACGCAAGTTTAATTACAGACCTGAA
>AB000263 |acc=AB000263|descr=Homo sapiens mRNA 
ACAAGATGCCATTGTCCCCCGGCCTCCTGCTGCTGCTGCTCTCCGGGGCCACGGCCACCGCTGCCCTGCCCCTGGAGGGT
GGCCCCACCGGCCGAGACAGCGAGCATATGCAGGAAGCGGCAGGAATAAGGAAAAGCAGCCTCCTGACTTTCCTCGCTTG
GTGGTTTGAGTGGACCTCCCAGGCCAGTGCCGGGCCCCTCATAGGAGAGGAAGCTCGGGAGGTGGCCAGGCGGCAGGAAG
GCGCACCCCCCCAGCAATCCGCGCGCCGGGACAGAATGCCCTGCAGGAACTTCTTCTGGAAGACCTTCTCCTCCTGCAAA
TAAAACCTCACCCATGAATGCTCACGCAAGTTTAATTACAGACCTGAA
\end{lstlisting}

\subsection*{Output}
The output of the \texttt{goose-fastaextractbyread} program is FASTA or Multi-FASTA file wiht the extracted sequences.\\
An example, using the value 0 as extraction starting point and the 50 as the end, for the provided input, is:
\begin{lstlisting}
>AB000264 |acc=AB000264|descr=Homo sapiens mRNA 
ACAAGACGGCCTCCTGCTGCTGCTGCTCTCCGGGGCCACGGCCCTGGAGG
>AB000263 |acc=AB000263|descr=Homo sapiens mRNA 
ACAAGATGCCATTGTCCCCCGGCCTCCTGCTGCTGCTGCTCTCCGGGGCC
\end{lstlisting}
\section{Program goose-fastainfo}
The \texttt{goose-fastainfo} shows the readed information of a FASTA or Multi-FASTA file format.\\
For help type:
\begin{lstlisting}
./goose-fastainfo -h
\end{lstlisting}
In the following subsections, we explain the input and output paramters.

\subsection{Input parameters}

The \texttt{goose-fastainfo} program needs two streams for the computation,
namely the input and output standard. The input stream is a FASTA or Multi-FASTA file.\\
The attribution is given according to:
\begin{lstlisting}
Usage: ./goose-fastainfo [options] [[--] args]
   or: ./goose-fastainfo [options]

It shows read information of a FASTA or Multi-FASTA file format.

    -h, --help            show this help message and exit

Basic options
    < input.fasta         Input FASTA or Multi-FASTA file format (stdin)
    > output              Output read information (stdout)

Example: ./goose-fastainfo < input.fasta > output
\end{lstlisting}
An example on such an input file is:
\begin{lstlisting}
>AB000264 |acc=AB000264|descr=Homo sapiens mRNA 
ACAAGACGGCCTCCTGCTGCTGCTGCTCTCCGGGGCCACGGCCCTGGAGGGTCCACCGCTGCCCTGCTGCCATTGTCCCC
GGCCCCACCTAAGGAAAAGCAGCCTCCTGACTTTCCTCGCTTGGGCCGAGACAGCGAGCATATGCAGGAAGCGGCAGGAA
GTGGTTTGAGTGGACCTCCGGGCCCCTCATAGGAGAGGAAGCTCGGGAGGTGGCCAGGCGGCAGGAAGCAGGCCAGTGCC
GCGAATCCGCGCGCCGGGACAGAATCTCCTGCAAAGCCCTGCAGGAACTTCTTCTGGAAGACCTTCTCCACCCCCCCAGC
TAAAACCTCACCCATGAATGCTCACGCAAGTTTAATTACAGACCTGAA
>AB000263 |acc=AB000263|descr=Homo sapiens mRNA 
ACAAGATGCCATTGTCCCCCGGCCTCCTGCTGCTGCTGCTCTCCGGGGCCACGGCCACCGCTGCCCTGCCCCTGGAGGGT
GGCCCCACCGGCCGAGACAGCGAGCATATGCAGGAAGCGGCAGGAATAAGGAAAAGCAGCCTCCTGACTTTCCTCGCTTG
GTGGTTTGAGTGGACCTCCCAGGCCAGTGCCGGGCCCCTCATAGGAGAGGAAGCTCGGGAGGTGGCCAGGCGGCAGGAAG
GCGCACCCCCCCAGCAATCCGCGCGCCGGGACAGAATGCCCTGCAGGAACTTCTTCTGGAAGACCTTCTCCTCCTGCAAA
TAAAACCTCACCCATGAATGCTCACGCAAGTTTAATTACAGACCTGAA
\end{lstlisting}

\subsection{Output}
The output of the \texttt{goose-fastainfo} program is a set of informations related with the file readed. \\
An example, for the input, is:
\begin{lstlisting}
Number of reads      : 2
Number of bases      : 736
MIN of bases in read : 368
MAX of bases in read : 368
AVG of bases in read : 368.0000
\end{lstlisting} 
\section{Program goose-mutatefasta}
The \texttt{goose-mutatefasta} creates a synthetic mutation of a fasta file given specific rates of editions, deletions and additions. All these paramenters are defined by the user, and their are optional.\\
For help type:
\begin{lstlisting}
./goose-mutatefasta -h
\end{lstlisting}
In the following subsections, we explain the input and output paramters.

\subsection{Input parameters}

The \texttt{goose-mutatefasta} program needs two streams for the computation, namely the input and output standard. However, optional settings can be supplied too, such as the starting point to the random generator, and the edition, deletion and insertion rates. Also, the user can choose to use the ACGTN alphabet in the synthetic mutation. The input stream is a FASTA or Multi-FASTA File.\\
The attribution is given according to:
\begin{lstlisting}
Usage: ./goose-mutatefasta [options] [[--] args]
   or: ./goose-mutatefasta [options]

Creates a synthetic mutation of a fasta file given specific rates of editions, deletions and additions

    -h, --help                    show this help message and exit

Basic options
    < input.fasta                 Input FASTA or Multi-FASTA file format (stdin)
    > output.fasta                Output FASTA or Multi-FASTA file format (stdout)

Optional
    -s, --seed=<int>              Starting point to the random generator
    -e, --edit-rate=<dbl>         Defines the edition rate (default 0.0)
    -d, --deletion-rate=<dbl>     Defines the deletion rate (default 0.0)
    -i, --insertion-rate=<dbl>    Defines the insertion rate (default 0.0)
    -a, --ACGTN-alphabet          When active, the application uses the ACGTN alphabet

Example: ./goose-mutatefasta -s <seed> -e <edit rate> -d <deletion rate> -i <insertion rate> -a < input.fasta > output.fasta
\end{lstlisting}

An example on such an input file is:
\begin{lstlisting}
>AB000264 |acc=AB000264|descr=Homo sapiens mRNA 
ACAAGACGGCCTCCTGCTGCTGCTGCTCTCCGGGGCCACGGCCCTGGAGGGTCCACCGCTGCCCTGCTGCCATTGTCCCC
GGCCCCACCTAAGGAAAAGCAGCCTCCTGACTTTCCTCGCTTGGGCCGAGACAGCGAGCATATGCAGGAAGCGGCAGGAA
GTGGTTTGAGTGGACCTCCGGGCCCCTCATAGGAGAGGAAGCTCGGGAGGTGGCCAGGCGGCAGGAAGCAGGCCAGTGCC
GCGAATCCGCGCGCCGGGACAGAATCTCCTGCAAAGCCCTGCAGGAACTTCTTCTGGAAGACCTTCTCCACCCCCCCAGC
TAAAACCTCACCCATGAATGCTCACGCAAGTTTAATTACAGACCTGAA
>AB000263 |acc=AB000263|descr=Homo sapiens mRNA 
ACAAGATGCCATTGTCCCCCGGCCTCCTGCTGCTGCTGCTCTCCGGGGCCACGGCCACCGCTGCCCTGCCCCTGGAGGGT
GGCCCCACCGGCCGAGACAGCGAGCATATGCAGGAAGCGGCAGGAATAAGGAAAAGCAGCCTCCTGACTTTCCTCGCTTG
GTGGTTTGAGTGGACCTCCCAGGCCAGTGCCGGGCCCCTCATAGGAGAGGAAGCTCGGGAGGTGGCCAGGCGGCAGGAAG
GCGCACCCCCCCAGCAATCCGCGCGCCGGGACAGAATGCCCTGCAGGAACTTCTTCTGGAAGACCTTCTCCTCCTGCAAA
TAAAACCTCACCCATGAATGCTCACGCAAGTTTAATTACAGACCTGAA
\end{lstlisting}

\subsection{Output}
The output of the \texttt{goose-mutatefasta} program is a FASTA or Multi-FASTA file whith the synthetic mutation of input file.\\
Using the seed value as 1 and the edition rate as 0.5, an example for this input, is: 
\begin{lstlisting}
>AB000264 |acc=AB000264|descr=Homo sapiens mRNA 
ACGCAACGNATTCCTGCTGATCATANTGTNCCGCNCCCCNGCGACGGGGNCTCNCNNGCACACATNGTACCATTGTCCAC
NCTTNCANGTNANCGCTAGCAGGCTACNGTTTNTCCTCNCCTANNCCAANCNGGCGTNNNTACACTGGCACGTGCAGGCA
TNGGTCGGCNGGNNCCTCCGGNAACGGCACCGGAGACGAAGCTCGGNGGNTATACAGGTGTCANGAAACATCCCCGCGNC
GNGTGNCCNNGAANCCANAGAGTATCTCACTCACAACCCTGCGTGCACNTCTAGAGNANGACCTTACNCACCNTCCCNTT
NNGTACCACACCAATGAACGCTGCAGAAAGTCTGTTTNNAGGNGNGCA
>AB000263 |acc=AB000263|descr=Homo sapiens mRNA 
ATTTGAAGGCAANCGGNCCAGNAATNCGGNGGGTGCNGCTCNTGTNGGCTACGGNCATCGCGGCCCTGCTNTANTAAGCN
TGAACCACCGNTCGNNGCACTTAGCAATNGCGNAANCCGTCGGCACGGCGGAGACNAANCCGCTANTNNTTTCCCGCTNA
ATGGNTGTACAAGACCNACTANACCANCCTCCGTCACCACACTGGAGCGCANGATGGNNCGCTGNCTAGNAGNCNNTGAG
GCGCTCCNTCCTANAAANCCGTGGNCGAGCNCCCTATGGNAGNGTGGGGGTTTTACCGGAAGACCNTCGNGCCCTATGGG
AGCAATCANAANCTAGAAAGCTTACNGATGGTGANGAANTAGACTANG
\end{lstlisting}
\section{Program goose-randfastaextrachars}
The \texttt{goose-randfastaextrachars} substitues in the DNA sequence the outside ACGT chars by random ACGT symbols. It works both in FASTA and Multi-FASTA file formats.\\
For help type:
\begin{lstlisting}
./goose-randfastaextrachars -h
\end{lstlisting}
In the following subsections, we explain the input and output paramters.

\subsection*{Input parameters}

The \texttt{goose-randfastaextrachars} program needs two streams for the computation,
namely the input and output standard. The input stream is a FASTA or Multi-FASTA file.\\
The attribution is given according to:
\begin{lstlisting}
Usage: ./goose-randfastaextrachars [options] [[--] args]
   or: ./goose-randfastaextrachars [options]

It substitues in the DNA sequence the outside ACGT chars by random ACGT symbols.
It works both in FASTA and Multi-FASTA file formats


    -h, --help            show this help message and exit

Basic options
    < input.fasta         Input FASTA or Multi-FASTA file format (stdin)
    > output.fasta        Output FASTA or Multi-FASTA file format (stdout)

Example: ./goose-randfastaextrachars < input.fasta > output.fasta
\end{lstlisting}

An example on such an input file is:
\begin{lstlisting}
to do
\end{lstlisting}

\subsection*{Output}
The output of the \texttt{goose-randfastaextrachars} program is a FASTA or Multi-FASTA file.\\
An example, for the input, is:
\begin{lstlisting}
to do
\end{lstlisting}
\section{Program goose-extractreadbypattern}
The \texttt{goose-extractreadbypattern} ...

For help type:
\begin{lstlisting}
./goose-extractreadbypattern -h
\end{lstlisting}
In the following subsections, we explain the input and output paramters.

\subsection*{Input parameters}

The \texttt{goose-extractreadbypattern} program needs ...
The attribution is given according to:
\begin{lstlisting}
TO DO
\end{lstlisting}

An example on such an input file is:
\begin{lstlisting}
TO DO
\end{lstlisting}

\subsection*{Output}
The output of the \texttt{goose-extractreadbypattern} program ...
An example, for the input, is:
\begin{lstlisting}
TO DO
\end{lstlisting}

\section{Program goose-findnpos}
The \texttt{goose-findnpos} ...

For help type:
\begin{lstlisting}
./goose-findnpos -h
\end{lstlisting}
In the following subsections, we explain the input and output paramters.

\subsection*{Input parameters}

The \texttt{goose-findnpos} program needs ...
The attribution is given according to:
\begin{lstlisting}
TO DO
\end{lstlisting}

An example on such an input file is:
\begin{lstlisting}
TO DO
\end{lstlisting}

\subsection*{Output}
The output of the \texttt{goose-findnpos} program ...
An example, for the input, is:
\begin{lstlisting}
TO DO
\end{lstlisting}

\section{Program goose-seq2fasta}
The \texttt{goose-seq2fasta} converts a genomic sequence to pseudo FASTA file format.\\
For help type:
\begin{lstlisting}
./goose-seq2fasta -h
\end{lstlisting}
In the following subsections, we explain the input and output paramters.

\subsection*{Input parameters}

The \texttt{goose-seq2fasta} program needs two streams for the computation,
namely the input and output standard. The input stream is a sequence group file.\\
The attribution is given according to:
\begin{lstlisting}
Usage: ./goose-seq2fasta [options] [[--] args]
   or: ./goose-seq2fasta [options]

It converts a genomic sequence to pseudo FASTA file format.

    -h, --help            show this help message and exit

Basic options
    < input.fa            Input Multi-FASTA file format (stdin)
    > output.fa           Output Multi-FASTA file format (stdout)

Optional options
    -n, --name=<str>      The read's header
    -l, --lineSize=<int>  The maximum of chars for line

Example: ./goose-seq2fasta -l <lineSize> -n <name> < input.seq > output.fa
\end{lstlisting}
An example on such an input file is:
\begin{lstlisting}
ACAAGACGGCCTCCTGCTGCTGCTGCTCTCCGGGGCCACGGCCCTGGAGGGTCCACCGCTGCCCTGCTGCCATTGTCCCC
GGCCCCACCTAAGGAAAAGCAGCCTCCTGACTTTCCTCGCTTGGGCCGAGACAGCGAGCATATGCAGGAAGCGGCAGGAA
GTGGTTTGAGTGGACCTCCGGGCCCCTCATAGGAGAGGAAGCTCGGGAGGTGGCCAGGCGGCAGGAAGCAGGCCAGTGCC
GCGAATCCGCGCGCCGGGACAGAATCTCCTGCAAAGCCCTGCAGGAACTTCTTCTGGAAGACCTTCTCCACCCCCCCAGC
TAAAACCTCACCCATGAATGCTCACGCAAGTTTAATTACAGACCTGAAACAAGATGCCATTGTCCCCCGGCCTCCTGCTG
CTGCTGCTCTCCGGGGCCACGGCCACCGCTGCCCTGCCCCTGGAGGGTGGCCCCACCGGCCGAGACAGCGAGCATATGCA
GGAAGCGGCAGGAATAAGGAAAAGCAGCCTCCTGACTTTCCTCGCTTGGTGGTTTGAGTGGACCTCCCAGGCCAGTGCCG
GGCCCCTCATAGGAGAGGAAGCTCGGGAGGTGGCCAGGCGGCAGGAAGGCGCACCCCCCCAGCAATCCGCGCGCCGGGAC
AGAATGCCCTGCAGGAACTTCTTCTGGAAGACCTTCTCCTCCTGCAAATAAAACCTCACCCATGAATGCTCACGCAAGTT
TAATTACAGACCTGAA
\end{lstlisting}

\subsection*{Output}
The output of the \texttt{goose-seq2fasta} program is a pseudo FASTA file.\\
An example, using the size line as 80 and the read's header as ''Seq2Fasta'', for the input, is:
\begin{lstlisting}
>Seq2Fasta
ACAAGACGGCCTCCTGCTGCTGCTGCTCTCCGGGGCCACGGCCCTGGAGGGTCCACCGCTGCCCTGCTGCCATTGTCCCC
GGCCCCACCTAAGGAAAAGCAGCCTCCTGACTTTCCTCGCTTGGGCCGAGACAGCGAGCATATGCAGGAAGCGGCAGGAA
GTGGTTTGAGTGGACCTCCGGGCCCCTCATAGGAGAGGAAGCTCGGGAGGTGGCCAGGCGGCAGGAAGCAGGCCAGTGCC
GCGAATCCGCGCGCCGGGACAGAATCTCCTGCAAAGCCCTGCAGGAACTTCTTCTGGAAGACCTTCTCCACCCCCCCAGC
TAAAACCTCACCCATGAATGCTCACGCAAGTTTAATTACAGACCTGAAACAAGATGCCATTGTCCCCCGGCCTCCTGCTG
CTGCTGCTCTCCGGGGCCACGGCCACCGCTGCCCTGCCCCTGGAGGGTGGCCCCACCGGCCGAGACAGCGAGCATATGCA
GGAAGCGGCAGGAATAAGGAAAAGCAGCCTCCTGACTTTCCTCGCTTGGTGGTTTGAGTGGACCTCCCAGGCCAGTGCCG
GGCCCCTCATAGGAGAGGAAGCTCGGGAGGTGGCCAGGCGGCAGGAAGGCGCACCCCCCCAGCAATCCGCGCGCCGGGAC
AGAATGCCCTGCAGGAACTTCTTCTGGAAGACCTTCTCCTCCTGCAAATAAAACCTCACCCATGAATGCTCACGCAAGTT
TAATTACAGACCTGAA
\end{lstlisting}

%\section{Program goose-permuteseqbyblocks}
The \texttt{goose-permuteseqbyblocks} ...

For help type:
\begin{lstlisting}
./goose-permuteseqbyblocks -h
\end{lstlisting}
In the following subsections, we explain the input and output paramters.

\subsection{Input parameters}

The \texttt{goose-permuteseqbyblocks} program needs ...
The attribution is given according to:
\begin{lstlisting}
TO DO
\end{lstlisting}

An example on such an input file is:
\begin{lstlisting}
TO DO
\end{lstlisting}

\subsection{Output}
The output of the \texttt{goose-permuteseqbyblocks} program ...
An example, for the input, is:
\begin{lstlisting}
TO DO
\end{lstlisting}

\section{Program goose-splitreads}
The \texttt{goose-splitreads} splits a Multi-FASTA file to multiple FASTA files.\\
For help type:
\begin{lstlisting}
./goose-splitreads -h
\end{lstlisting}
In the following subsections, we explain the input and output paramters.

\subsection*{Input parameters}

The \texttt{goose-splitreads} program needs one stream for the computation,
namely the input standard. This input stream is a Multi-FASTA file.\\
The attribution is given according to:
\begin{lstlisting}
Usage: ./goose-splitreads [options] [[--] args]
   or: ./goose-splitreads [options]

It splits a Multi-FASTA file to multiple FASTA files.

    -h, --help            show this help message and exit

Basic options
    < input.fa            Input Multi-FASTA file format (stdin)

Optional options
    -l, --location=<str>  Location to store the files

Example: ./goose-splitreads < input.fasta
\end{lstlisting}
An example on such an input file is:
\begin{lstlisting}
>AB000264 |acc=AB000264|descr=Homo sapiens mRNA 
ACAAGACGGCCTCCTGCTGCTGCTGCTCTCCGGGGCCACGGCCCTGGAGGGTCCACCGCTGCCCTGCTGCCATTGTCCCC
GGCCCCACCTAAGGAAAAGCAGCCTCCTGACTTTCCTCGCTTGGGCCGAGACAGCGAGCATATGCAGGAAGCGGCAGGAA
GTGGTTTGAGTGGACCTCCGGGCCCCTCATAGGAGAGGAAGCTCGGGAGGTGGCCAGGCGGCAGGAAGCAGGCCAGTGCC
GCGAATCCGCGCGCCGGGACAGAATCTCCTGCAAAGCCCTGCAGGAACTTCTTCTGGAAGACCTTCTCCACCCCCCCAGC
TAAAACCTCACCCATGAATGCTCACGCAAGTTTAATTACAGACCTGAA
>AB000263 |acc=AB000263|descr=Homo sapiens mRNA 
ACAAGATGCCATTGTCCCCCGGCCTCCTGCTGCTGCTGCTCTCCGGGGCCACGGCCACCGCTGCCCTGCCCCTGGAGGGT
GGCCCCACCGGCCGAGACAGCGAGCATATGCAGGAAGCGGCAGGAATAAGGAAAAGCAGCCTCCTGACTTTCCTCGCTTG
GTGGTTTGAGTGGACCTCCCAGGCCAGTGCCGGGCCCCTCATAGGAGAGGAAGCTCGGGAGGTGGCCAGGCGGCAGGAAG
GCGCACCCCCCCAGCAATCCGCGCGCCGGGACAGAATGCCCTGCAGGAACTTCTTCTGGAAGACCTTCTCCTCCTGCAAA
TAAAACCTCACCCATGAATGCTCACGCAAGTTTAATTACAGACCTGAA
\end{lstlisting}

\subsection*{Output}
The output of the \texttt{goose-splitreads} program is a report summary of the execution, and the files created in the defined location.\\ 
An example, for the input, is:
\begin{lstlisting}
1 : Splitting to file:./out1.fa
2 : Splitting to file:./out2.fa
\end{lstlisting}



%Incomplete tools
%%\section{Program goose-reverselm}
The \texttt{goose-reverselm} ...

For help type:
\begin{lstlisting}
./goose-reverselm -h
\end{lstlisting}
In the following subsections, we explain the input and output paramters.

\subsection{Input parameters}

The \texttt{goose-reverselm} program needs ...
The attribution is given according to:
\begin{lstlisting}
TO DO
\end{lstlisting}

An example on such an input file is:
\begin{lstlisting}
TO DO
\end{lstlisting}

\subsection{Output}
The output of the \texttt{goose-reverselm} program ...
An example, for the input, is:
\begin{lstlisting}
TO DO
\end{lstlisting}