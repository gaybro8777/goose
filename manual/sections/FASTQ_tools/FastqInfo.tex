\section{Program goose-FastqInfo}
The \texttt{goose-FastqInfo} analyses the basic informations of FASTQ file format.\\
For help type:
\begin{lstlisting}
./goose-FastqInfo -h
\end{lstlisting}
In the following subsections, we explain the input and output paramters.

\subsection*{Input parameters}

The \texttt{goose-FastqInfo} program needs two streams for the computation,
namely the input and output standard. The input stream is a FASTQ file.\\
The attribution is given according to:
\begin{lstlisting}
Usage: ./goose-FastqInfo [options] [[--] args]
   or: ./goose-FastqInfo [options]

It analyses the basic informations of FASTQ file format.

    -h, --help            show this help message and exit

Basic options
    < input.fastq         Input FASTQ file format (stdin)
    > output              Output read information (stdout)

Example: ./goose-FastqInfo < input.fastq > output

Output example :
Total reads     : value
Max read length : value
Min read length : value
Min QS value    : value
Max QS value    : value
QS range        : value
\end{lstlisting}
An example on such an input file is:
\begin{lstlisting}
@SRR001666.1 071112_SLXA-EAS1_s_7:5:1:817:345 length=72
GGGTGATGGCCGCTGCCGATGGCGTCAAATCCCACCAAGTTACCCTTAACAACTTAAGGGTTTTCAAATAGA
+SRR001666.1 071112_SLXA-EAS1_s_7:5:1:817:345 length=72
IIIIIIIIIIIIIIIIIIIIIIIIIIIIII9IG9ICIIIIIIIIIIIIIIIIIIIIDIIIIIII>IIIIII/
@SRR001666.2 071112_SLXA-EAS1_s_7:5:1:801:338 length=72
GTTCAGGGATACGACGTTTGTATTTTAAGAATCTGAAGCAGAAGTCGATGATAATACGCGTCGTTTTATCAT
+SRR001666.2 071112_SLXA-EAS1_s_7:5:1:801:338 length=72
IIIIIIIIIIIIIIIIIIIIIIIIIIIIIIII6IBIIIIIIIIIIIIIIIIIIIIIIIGII>IIIII-I)8I
\end{lstlisting}

\subsection*{Output}
The output of the \texttt{goose-FastqInfo} program is a set of informations related with the file readed. \\
An example, for the input, is:
\begin{lstlisting}
Total reads     : 2
Max read length : 72
Min read length : 72
Min QS value    : 41
Max QS value    : 73
QS range        : 33
\end{lstlisting}
