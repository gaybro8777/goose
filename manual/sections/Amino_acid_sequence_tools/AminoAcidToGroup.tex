\section{Program goose-AminoAcidToGroup}

The \texttt{goose-AminoAcidToGroup} converts an amino acid sequence to a group 
sequence.\\
For help type:
\begin{lstlisting}
./goose-AminoAcidToGroup -h
\end{lstlisting}
In the following subsections, we explain the input and output paramters.

\subsection{Input parameters}

The \texttt{goose-AminoAcidToGroup} program needs two streams for the computation,
namely the input and output standard. The input stream is an amino acid sequence.
The attribution is given according to:
\begin{lstlisting}
Usage: ./goose-AminoAcidToGroup < in.prot > out.group
It converts a amino acid sequence to a group sequence.
Table:
Prot	Group
R	P
H	P  Amino acids with electric charged side chains: POSITIVE
K	P
-	-
D	N
E	N  Amino acids with electric charged side chains: NEGATIVE
-	-
S	U
T	U
N	U  Amino acids with electric UNCHARGED side chains
Q	U
-	-
C	S
U	S
G	S  Special cases
P	S
-	-
A	H
V	H
I	H
L	H
M	H  Amino acids with hydrophobic side chains
F	H
Y	H
W	H
-	-
*	*  Others
X	X  Unknown
\end{lstlisting}
It can be used to group amino acids by properties, such as electric charge (positive
and negative), uncharged side chains, hydrophobic side chains and special cases.
An example on such an input file is:
\begin{lstlisting}
IPFLLKKQFALADKLVLSKLRQLLGGRIKMMPCGGAKLEPAIGLFFHAIGINIKLGYGMTETTATVSCWHDFQFNPNSIG
TLMPKAEVKIGENNEILVRGGMVMKGYYKKPEETAQAFTEDGFLKTGDAGEFDEQGNLFITDRIKELMKTSNGKYIAPQY
IESKIGKDKFIEQIAIIADAKKYVSALIVPCFDSLEEYAKQLNIKYHDRLELLKNSDILKMFE
\end{lstlisting}

\subsection{Output}

The output of the \texttt{goose-AminoAcidToGroup} program is a group sequence.\\
An example, for the input, is:
\begin{lstlisting}
HSHHHPPUHHHHNPHHHUPHPUHHSSPHPHHSSSSHPHNSHHSHHHPHHSHUHPHSHSHUNUUHUHUSHPNHUHUSUUHS
UHHSPHNHPHSNUUNHHHPSSHHHPSHHPPSNNUHUHHUNNSHHPUSNHSNHNNUSUHHHUNPHPNHHPUUUSPHHHSUH
HNUPHSPNPHHNUHHHHHNHPPHHUHHHHSSHNUHNNHHPUHUHPHPNPHNHHPUUNHHPHHN
\end{lstlisting}
